\documentclass[10pt]{article}
\usepackage{float}
\usepackage{amsmath}
\usepackage{graphicx}
\input{epsf}
%\usepackage{a4}

\newtheorem{theorem}{Theorem}
\newtheorem{acknowledgement}[theorem]{Acknowledgement}
\newtheorem{algorithm}[theorem]{Algorithm}
\newtheorem{axiom}[theorem]{Axiom}
\newtheorem{case}[theorem]{Case}
\newtheorem{claim}[theorem]{Claim}
\newtheorem{conclusion}[theorem]{Conclusion}
\newtheorem{condition}[theorem]{Condition}
\newtheorem{conjecture}[theorem]{Conjecture}
\newtheorem{corollary}[theorem]{Corollary}
\newtheorem{criterion}[theorem]{Criterion}
\newtheorem{definition}[theorem]{Definition}
\newtheorem{example}[theorem]{Example}
\newtheorem{exercise}[theorem]{Exercise}
\newtheorem{lemma}[theorem]{Lemma}
\newtheorem{notation}[theorem]{Notation}
\newtheorem{problem}[theorem]{Problem}
\newtheorem{proposition}[theorem]{Proposition}
\newtheorem{remark}[theorem]{Remark}
\newtheorem{solution}[theorem]{Solution}
\newtheorem{summary}[theorem]{Summary}
\newenvironment{proof}[1][Proof]{\textbf{#1.} }{\ \rule{0.5em}{0.5em}}
%\input{tcilatex}

%FXG Commands
\newtheorem{guess}{Definition}
%\newcommand{\diff2}[2] {\frac{\partial^2 #1}{ \partial {#2}^2}}
%\newcommand{\diff2}[2] {\frac{\partial #1}{\partial #2}}
\newcommand{\Norder} {N}
\newcommand{\order}{\mathcal{O}}
\newcommand{\Npoints} {N_p}
\newcommand{\diff}[2] {\frac{\partial #1}{\partial #2}}
\newcommand{\dxx}[2] {\frac{\partial^2 #1}{\partial {#2}^2}}
\newcommand{\difft}[2] {\frac{d #1}{d #2}}
\newcommand{\lagrange}[1] {\frac{d #1}{dt}}
\newcommand{\lebesgue}{\parallel I \parallel}
\newcommand{\polysp}{\mathcal{P}_N}
\newcommand{\vc}[1]{\mbox{\boldmath$#1$\unboldmath}}
\newcommand{\grad}{\vc{\nabla}}
\newcommand{\inte}{\int_{\mbox{\footnotesize ${\Omega_e}$}}}
\newcommand{\intce}{\int_{\mbox{\footnotesize ${\widehat{\Omega}_e}$}}}
\newcommand{\intb}{\int_{\mbox{\footnotesize ${\Gamma_e}$}}}
\newcommand{\intcb}{\int_{\mbox{\footnotesize ${\widehat{\Gamma}_e}$}}}
\newcommand{\inth}{\int_{\mbox{\footnotesize ${\Omega}$}}}
\newcommand{\inthb}{\int_{\mbox{\footnotesize ${\Gamma}$}}}
\newcommand{\intv}{\int_{\mbox{\footnotesize ${\sigma}$}}}
\newcommand{\sumv}{\sum_{K=1}^{N_{\mathrm{lev}}}}
\newcommand{\sumk}{\sum_{L=1}^{K}}
\newcommand{\half}{\frac{1}{2}}
\newcommand{\inti}{\int_{\mbox{\footnotesize\sf I}}}
\newcommand{\intbd}{\oint_{\mbox{\footnotesize ${\delta}$\sf D}}}
\newcommand{\intbi}{\oint_{\mbox{\footnotesize ${\delta}$\sf I}}}
\newcommand{\ldnorm}[1]{\left\| #1 \right\|_{\mbox{\footnotesize \sf D}} }
\newcommand{\lonorm}[1]{\left\| #1 \right\|_{\Omega}}
\newcommand{\spc}[1]{\mbox{\sf #1}}
\newcommand{\ope}[1]{{\cal #1}}
\newcommand{\mt}[1]{{\rm #1}}
\newcommand{\dis}{\displaystyle}
\newcommand{\ve}{\varepsilon}
\newcommand{\ov}{\overline}
\newcommand{\wt}{\widetilde}
\newcommand{\wh}{\widehat}
\newcommand{\be}{\begin{equation}}
\newcommand{\ee}{\end{equation}}
\def\bepsilon{\mbox{\boldmath $\epsilon $}}
\def\bpsi{\mbox{\boldmath $\psi $}}
\def\bphi{\mbox{\boldmath $\phi $}}
\def\bmu{\mbox{\boldmath $\mu $}}
\def\Et{ \tilde{E} }
\def\Ht{ \tilde{H} }
\def\sdot{ \dot{\sigma} }
\newcommand{\innerd}[2]{\left( #1,#2 \right)_{\mbox{\footnotesize \sf D}}}
\newcommand{\inners}[2]{\left( #1,#2 \right)_{\mbox{\footnotesize
${\delta}$\sf D}}}
\newcommand{\innerbd}[2]{\left( #1,#2 \right)_{\mbox{\footnotesize ${\delta}$\sf
 D}}}
\newcommand{\innerO}[2]{\left( #1,#2 \right)_{\Omega}}
\newcommand{\innerOs}[2]{\left( #1,#2 \right)_{\delta \Omega}}
\newcommand{\innerdk}[2]{\left( #1,#2 \right)_{\mbox{\footnotesize \sf D}^k}}
\newcommand{\intbdk}{\oint_{\mbox{\footnotesize ${\delta}$\sf D}^k}}
\newcommand{\ldnormk}[1]{\left\| #1 \right\|_{\mbox{\footnotesize \sf D}^k}}
\newcommand{\intdk}{\int_{\mbox{\footnotesize \sf D}^k}}
\newcommand{\epsD}{\varepsilon_{\mbox{\footnotesize \sf D}}}
\newcommand{\ldnormsob}[2]{\left\| #2 \right\|_{W^{#1}(\mbox{\footnotesize \sf D
})}}
\newcommand{\lbdnorm}[1]{\left\| #1 \right\|_{\mbox{\footnotesize \sf $\delta$D}
}}
\renewcommand{\thetable}{\Roman{table}}
\newcommand{\qvector}{\vc{q}}

\DeclareMathOperator{\Span}{span}
\DeclareMathOperator{\Dim}{dim}

\newcommand{\polyquad}{\mathcal{Q}_{N}}
\newcommand{\polyP}{\mathcal{P}_{N}}
\newcommand{\polyPnpm}{\mathcal{P}_{(N+M)}}
\newcommand{\polyPd}{\mathcal{P}_{d}}
\newcommand{\polyPnm}{\mathcal{P}_{N,M}}
\newcommand{\polyPn}{\mathcal{P}_{N,0}}
\newcommand{\transpose}{^{\mathcal{T}}}

\begin{document}
\title{MA4245 Mathematical Principles of Galerkin Methods \\
Project 4: 2D Wave Equation}
\author{Prof. Frank Giraldo \\
Department of Applied Mathematics \\
Naval Postgraduate School \\
Monterey, CA 93943-5216}
\date{Due on Friday June 11 at 5pm}

\maketitle


\section{Continuous Problem}
The governing partial differential equation (PDE) is
\[
\diff{q}{t} + \vc{u} \cdot \grad q = 0 \qquad \forall (x,y) \in [-1,1]^2
\]
where $q=q(x,y,t)$ and $\vc{u}=\vc{u}(x,y)$ with $\vc{u}=(u,v)^T$. Let the velocity field be 
\[
u(x,y)=y \qquad \mathrm{and} \qquad v(x,y)=-x
\]
which forms a velocity field that rotates fluid particles in a clockwise direction. Note that this velocity field is divergence free, that is, 
that the following condition is satisfied
\[
\grad \cdot \vc{u}=0.
\]

The reason why this condition is important is that the identity
\[
\grad \cdot ( q \vc{u} ) = \vc{u} \cdot \grad q + q \grad \cdot \vc{u} 
\]
simplifies to 
\[
\grad \cdot ( q \vc{u} ) = \vc{u} \cdot \grad q
\]
which means that we can rewrite the initial problem statement in the conservation form
\[
\diff{q}{t} + \grad \cdot \vc{f} = 0 \qquad \forall (x,y) \in [-1,1]^2
\]
where $\vc{f}=q \vc{u}$ is the flux. This form is now ready for use with the DG method.

Clearly, this problem represents a 2D wave equation that is hyperbolic and is thereby an initial value problem that requires an initial condition. 
Let that initial condition be the Gaussian
\[
q(x,y,0)= e ^{ - \sigma\left[ (x-x_c)^2 + (y-y_c)^2 \right] }
\]
where $(x_c,y_c)=(-0.5,0)$ is the initial center of the Gaussian and $\sigma=32$ controls the shape (steepness) of the Gaussian wave.
Use periodic boundary conditions for all four sides (i.e., you are solving flow on the surface of a torus; the iperiodic array will take care of this for you if you are using CG, for DG you have to do it via the fluxes).

\section{Simulations}
Use both CG AND DG methods to solve this equation using the same code base. I
recommend you try CG first and then move on to DG. For DG you will
need additional maps 
for the faces (given in the GitHub site). 
You need to write the code to handle arbitrarily-sized elements and polynomial orders, and use inexact integration formulas. What I mean by arbitrarily-sized elements is that you should not assume that each element is of the same size. You can, however, assume that each element uses the same polynomial and integration orders.

\subsection{Skeleton Code Provided}
In the GitHub site under Project 4, you will see code to help you get started with the project. The file "A$\_$Main$\_$Driver..." is the driver file.  Go there and see where it says "Students Add your Code here".  Open up "Construct$\_$RHS$\_$Vector" and see where you need to add code. You need to construct a RHS vector that includes both the contribution from the differentiation matrix and the flux matrix as discussed in the class lectures. 

\subsection{Streamline}
Once your code works, be sure to streamline it in order to make it as fast as possible - this is the fun stuff and take advantage of all you have learned. 

\subsection{Results You Need to Show}
You must show results for $N=1,2,4,8,16$ with increasing number of elements $N_e=nel^2$ where $nel$ denotes the number of quadrilaterals in 
the $x$ and $y$ directions. 
Plot broken $L^2$ error norms (defined below) versus number of points ($N_P$) and show all 5 curves on one plot. Remember that you must use 
a log plot for the error to capture the spectral convergence. 

For the following simulations, you must turn in four plots (one set of four for CG and another set of four for DG): two plots showing convergence rates (for inexact integration only) and another two plots showing 2-norm errors versus wallclock time. Write a discussion on your findings. 

Also, give me a complexity analysis of your code (operation count) to
see exactly how many operations your code is performing.

\paragraph{N=1 Simulations}
For $N=1$ use $nel=8, 16, 24, 32, 40, 48$ elements.

\paragraph{N=2 Simulations}
For $N=2$ use $nel=4, 8, 12, 16, 20, 24$ elements.

\paragraph{N=4 Simulations}
For $N=4$ use $nel=2, 4, 6, 8, 10, 12$ elements.

\paragraph{N=8 Simulations}
For $N=8$ use $nel=1, 2, 3, 4, 5, 6$ elements.

\paragraph{N=16 Simulations}
For $N=16$ use $nel=1,2,3$ elements.

Here is an example of the Convergence rates plot you should show me:
\begin{figure}[h]
\begin{center}
\begin{minipage}{2.20in}
\includegraphics[width=2.20in]{Matlab_Files/CG/figures/CG_exact_NonTensorProduct_NoFilter_ConvergenceRates.eps}
a) Exact Integration 
\end{minipage} \ \hspace{0.125in} \
\begin{minipage}{2.20in}
\includegraphics[width=2.20in]{Matlab_Files/CG/figures/CG_inexact_TensorProduct_NoFilter_ConvergenceRates.eps}
b) Inexact Integration
\end{minipage} 
\caption{The Convergence Rates for CG using a) exact and b) inexact integration using $C=0.25$ with RK3.}
\end{center}
\end{figure}

Here is an example of the Computational Cost plot you should show me:
\begin{figure}[h]
\begin{center}
\begin{minipage}{2.20in}
\includegraphics[width=2.20in]{Matlab_Files/CG/figures/CG_exact_NonTensorProduct_NoFilter_ConvergenceRates_CPU.eps}
a) Exact Integration 
\end{minipage} \ \hspace{0.125in} \
\begin{minipage}{2.20in}
\includegraphics[width=2.20in]{Matlab_Files/CG/figures/CG_inexact_TensorProduct_NoFilter_ConvergenceRates_CPU.eps}
b) Inexact Integration
\end{minipage} 
\caption{The Computational Cost for CG using a) exact and b) inexact integration using $C=0.25$ with RK3.}
\end{center}
\end{figure}



\section{Helpful Relations}
Make sure that your time-step $\Delta t$ is small enough to ensure stability. Recall that the Courant number
\[
C=u \frac{\Delta t}{\Delta x}
\]
must be within a certain value for stability. 
For the 4th order RK method I gave you, I used $C=0.5$ for all the simulations;
however, for stability purposes, the time-step can be much bigger.

\end{document}
